\chapter{യൂണിക്കോഡും മലയാളവും}
ലാടെൿ ഉപയോഗിച്ച് തയ്യാറാക്കുന്ന മലയാളം തീസിസ് റിപ്പോർട്ടിലെ അടുത്ത അദ്ധ്യായം.

\section{ചെറിയ തലക്കെട്ട്}

ഇത് ഒന്നാമത്തെ സെൿഷൻ.

\subsection{പിന്നെയും ചെറിയ തലക്കെട്ട്}
\paragraph{}രാപ്പകലുള്ള വീട്ടുവേല അന്തർജനങ്ങളുടെ ജീവിതയാഥാർത്ഥ്യമായിരുന്നു. ഇല്ലങ്ങളിൽ വലിയ സദ്യയും മറ്റും നടത്താൻ അന്തർജനങ്ങൾ പണിപ്പെടുന്നതിനെക്കുറിച്ച് 20-ാം നൂറ്റാണ്ടിന്റെ തുടക്കത്തിൽ ചില ലേഖകർ വിവരിച്ചിട്ടുണ്ട്. 1927ലെ നമ്പൂതിരി സ്ത്രീവിദ്യാഭ്യാസക്കമ്മിഷനു മുമ്പാകെ (നമ്പൂതിരിമാരുടെ സമുദായപ്രസ്ഥാനമായ യോഗക്ഷേമസഭയാണ് ഈ കമ്മിഷൻ സ്ഥാപിച്ചത്) തെളിവു നൽകിക്കൊണ്ട് നമ്പൂതിരിപെൺകുട്ടികളെക്കുറിച്ച് മാടമ്പ് നാരായണൻ നമ്പൂതിരി പറഞ്ഞതിങ്ങനെയാണ് \cite{devika}:

\begin{quotation}
 ``...അധികകാലം പഠിക്കുവാൻ അവർക്കു [അന്തർജനങ്ങൾക്ക്] സമയമില്ല. എട്ടുവയസ്സിൽ അടുക്കളപ്പണിയാരംഭിക്കുന്നു! അടുക്കള മെഴുകുക, പാത്രംതേക്കുക, എച്ചിലെടുക്കുക മുതലായതെല്ലാം അവരുടെ ജോലിതന്നെ. ഇതിന്നു പുറമെ അവരുടെ സഹോദരന്മാരെ `serve' ചെയ്യേണ്ടതും ഇവരുടെ മുറതന്നെ. ഇങ്ങനെ ഒന്നുരണ്ടുകൊല്ലം കഴിയുമ്പോഴേക്കും അതാ വേറെ ജോലിയുംകൂടി അവരുടെ തലയിൽ വച്ചുകെട്ടുന്നു. അതു മറ്റൊന്നുമല്ല; അന്തർജനഭാഷയിൽ `നേദിക്കുക'. രാവിലെ മുതൽ 10 മണിവരെ `നേദിക്കലും' `നമസ്ക്കാരവും' തന്നെ. കിഴക്കോട്ട്, തെക്കോട്ട്, എന്തിനേറെ എല്ലാ കോണിലേക്കും ഗുരുവായൂരപ്പനും കാവിൽ ശാസ്താവിനും വൈക്കത്തപ്പനും കോവിൽ അയ്യപ്പനും എന്നുവേണ്ട അവർക്കും ഇവർക്കും നേദിക്കൽതന്നെ. ഇതെല്ലാം ആജീവനാന്തം വേണ്ടതുമാണ്... പിന്നെ പകലെ ഉച്ചയ്ക്കുമേൽ രണ്ടുനാഴികയുള്ളതു പുരാണപാരായണത്തിനും 'ചരടുപിടിച്ചു ജപിക്കുന്ന'തിനും ആയി.'' \\
\\(നമ്പൂതിരി സ്ത്രീവിദ്യാഭ്യാസ കമ്മിഷൻ റിപ്പോർട്ട്, 1928, പുറം 65-66)
\end{quotation}
 
