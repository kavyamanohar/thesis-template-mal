 
\chapter{ആമുഖം}

ലാടെൿ ഉപയോഗിച്ച് തയ്യാറാക്കുന്ന മലയാളം തീസിസ് റിപ്പോർട്ടിന് ഒരു ആമുഖം.

\paragraph{}
 കേരളത്തിനും മലയാള ഭാഷയ്ക്കും ശ്രദ്ധേയമായ സംഭാവനകൾ നൽകിയ ജർമൻ ഭാഷാ പണ്ഡിതനായിരുന്നു റെവ്. ഡോ. ഹെർമൻ ഗുണ്ടർട്ട് (1814 ഫെബ്രുവരി 4 - 1893 ഏപ്രിൽ 25). ജർമനിയിലെ സ്റ്റുട്ട്ഗാർട്ട് എന്ന സ്ഥലത്ത് 1814 ഫെബ്രുവരി 4-നു ജനിച്ചു. 1836 ജൂലൈ 7-നു് ഇന്ത്യയിലെത്തി. മദ്രാസ് പ്രസിഡൻസിയുടെ വിവിധഭാഗങ്ങളിൽ മതപ്രചരണ സംബന്ധമായ ജോലികൾ നടത്തുന്നതിനിടയിൽ 1838 ഒക്ടോബർ 7-നു് ഗുണ്ടർട്ടും ഭാര്യയും തിരുനെൽവേലിയിൽ നിന്നും തിരുവന്തപുരത്തെത്തി താമസമാക്കി  \footnote{വിക്കിലേഖനം \url{ml.wikipedia.org}} .

\paragraph{}
തമിഴ്‌നാട്ടിലെ ഹ്രസ്വകാല ജീവിതത്തിനിടയിൽ തമിഴ്ഭാഷയിൽ പ്രസംഗപാടവം നേടിയ ഗുണ്ടർട്ട് അതിവേഗം മലയാളവും പഠിച്ചു.ഹെർമൻ ഗുണ്ടർട്ടിനെ മലയാളം പഠിപ്പിച്ചത് ഊരാച്ചേരി ഗുരുനാഥൻമാർ. തലശ്ശേരിക്കടുത്ത് ചൊക്ലിയിലെ കവിയൂർ ആണ് ഗുരുനാഥൻമാരുടെ ജന്മദേശം.ഇവരെക്കുറിച്ച് കേട്ടറിഞ്ഞ ഹെർമൻ ഗുണ്ടർട്ട് മലയാളം പഠിക്കാൻ ഇവരെ തേടിയെത്തുകയായിരുന്നു. താൻ താമസിച്ചിരുന്ന ഇല്ലിക്കുന്നിലേക്ക് ഊരാച്ചേരി ഗുരുനാഥൻമാരെ ക്ഷണിച്ചു കൊണ്ടുപോയായിരുന്നു ഗുണ്ടർട്ട് മലയാള ഭാഷയിൽ പ്രാവീണ്യം നേടിയത്. 

\paragraph{}
ബെഞ്ചമിൻ ബെയ്‌ലി മലയാളമുദ്രണം തുടങ്ങുന്ന കാലംമുതൽ മാദ്ധ്യമങ്ങളാണ് ഭാഷാപരിണാമത്തെ നയിച്ചത്. എന്നാൽ വ്യവസ്ഥാപിത മാദ്ധ്യമങ്ങളുടെ കൈയിൽ നിന്ന് ഇതിന്റെ നേതൃത്വം സാമൂഹ്യമാദ്ധ്യമങ്ങൾ ഏറ്റെടുക്കുന്ന കാഴ്ചയാണ് നാം കണ്ടുകൊണ്ടിരിക്കുന്നത്. ബെയ്‌ലി ഉപയോഗിച്ച മലയാളമോ അന്നത്തെ പ്രയോഗങ്ങളോ അല്ല ഇന്നു നിലവിലുള്ളത്. അക്ഷരങ്ങൾ പോലും മാറിപ്പോയിരിക്കുന്നു. കോട്ടയത്തുനിന്നു പ്രസിദ്ധീകരിക്കുന്ന ജ്ഞാനനിക്ഷേപത്തിൽ വന്ന ആനയും തുന്നാരനും കുറിച്ച എന്ന നാടോടിക്കഥയിലോ ഇല്ലിക്കുന്നിൽ നിന്ന് അച്ചടിച്ച ഒരു കല്ലൻ എന്ന ചെറുകഥയിലോ ആരംഭിക്കുന്നു, മലയാളത്തിലെ ഷോർട്ട് ഫിക്ഷൻ. അന്നുമുതലിന്നോളം മലയാളഭാഷയേയും സാഹിത്യത്തേയും രൂപപ്പെടുത്തുന്നതിൽ അച്ചടിമാദ്ധ്യമങ്ങളും അവ മെച്ചപ്പെടുന്നതിനായി ലഭ്യമായ പുതിയപുതിയ സാങ്കേതികവിദ്യകളും വലിയ പങ്കാണു വഹിച്ചത്. കാലാകാലങ്ങളിൽ അവ പല പുതിയ വാക്കുകളും പ്രയോഗങ്ങളും മാനകഭാഷയിലേക്കു ക്രമപ്പെടുത്തുകയും പലതിനെയും വഴിവക്കിൽ ഉപേക്ഷിക്കുകയും ചെയ്തു\cite{bailey}. 

\paragraph{}
തമിഴ്‌നാട്ടിലെ ഹ്രസ്വകാല ജീവിതത്തിനിടയിൽ തമിഴ്ഭാഷയിൽ പ്രസംഗപാടവം നേടിയ ഗുണ്ടർട്ട് അതിവേഗം മലയാളവും പഠിച്ചു.ഹെർമൻ ഗുണ്ടർട്ടിനെ മലയാളം പഠിപ്പിച്ചത് ഊരാച്ചേരി ഗുരുനാഥൻമാർ. തലശ്ശേരിക്കടുത്ത് ചൊക്ലിയിലെ കവിയൂർ ആണ് ഗുരുനാഥൻമാരുടെ ജന്മദേശം.ഇവരെക്കുറിച്ച് കേട്ടറിഞ്ഞ ഹെർമൻ ഗുണ്ടർട്ട് മലയാളം പഠിക്കാൻ ഇവരെ തേടിയെത്തുകയായിരുന്നു. താൻ താമസിച്ചിരുന്ന ഇല്ലിക്കുന്നിലേക്ക് ഊരാച്ചേരി ഗുരുനാഥൻമാരെ ക്ഷണിച്ചു കൊണ്ടുപോയായിരുന്നു ഗുണ്ടർട്ട് മലയാള ഭാഷയിൽ പ്രാവീണ്യം നേടിയത്. 

\paragraph{}
ബെഞ്ചമിൻ ബെയ്‌ലി മലയാളമുദ്രണം തുടങ്ങുന്ന കാലംമുതൽ മാദ്ധ്യമങ്ങളാണ് ഭാഷാപരിണാമത്തെ നയിച്ചത്. എന്നാൽ വ്യവസ്ഥാപിത മാദ്ധ്യമങ്ങളുടെ കൈയിൽ നിന്ന് ഇതിന്റെ നേതൃത്വം സാമൂഹ്യമാദ്ധ്യമങ്ങൾ ഏറ്റെടുക്കുന്ന കാഴ്ചയാണ് നാം കണ്ടുകൊണ്ടിരിക്കുന്നത്. ബെയ്‌ലി ഉപയോഗിച്ച മലയാളമോ അന്നത്തെ പ്രയോഗങ്ങളോ അല്ല ഇന്നു നിലവിലുള്ളത്. അക്ഷരങ്ങൾ പോലും മാറിപ്പോയിരിക്കുന്നു. കോട്ടയത്തുനിന്നു പ്രസിദ്ധീകരിക്കുന്ന ജ്ഞാനനിക്ഷേപത്തിൽ വന്ന ആനയും തുന്നാരനും കുറിച്ച എന്ന നാടോടിക്കഥയിലോ ഇല്ലിക്കുന്നിൽ നിന്ന് അച്ചടിച്ച ഒരു കല്ലൻ എന്ന ചെറുകഥയിലോ ആരംഭിക്കുന്നു, മലയാളത്തിലെ ഷോർട്ട് ഫിക്ഷൻ. അന്നുമുതലിന്നോളം മലയാളഭാഷയേയും സാഹിത്യത്തേയും രൂപപ്പെടുത്തുന്നതിൽ അച്ചടിമാദ്ധ്യമങ്ങളും അവ മെച്ചപ്പെടുന്നതിനായി ലഭ്യമായ പുതിയപുതിയ സാങ്കേതികവിദ്യകളും വലിയ പങ്കാണു വഹിച്ചത്. കാലാകാലങ്ങളിൽ അവ പല പുതിയ വാക്കുകളും പ്രയോഗങ്ങളും മാനകഭാഷയിലേക്കു ക്രമപ്പെടുത്തുകയും പലതിനെയും വഴിവക്കിൽ ഉപേക്ഷിക്കുകയും ചെയ്തു\cite{bailey}. 


\paragraph{}
 കേരളത്തിനും മലയാള ഭാഷയ്ക്കും ശ്രദ്ധേയമായ സംഭാവനകൾ നൽകിയ ജർമൻ ഭാഷാ പണ്ഡിതനായിരുന്നു റെവ്. ഡോ. ഹെർമൻ ഗുണ്ടർട്ട് (1814 ഫെബ്രുവരി 4 - 1893 ഏപ്രിൽ 25). ജർമനിയിലെ സ്റ്റുട്ട്ഗാർട്ട് എന്ന സ്ഥലത്ത് 1814 ഫെബ്രുവരി 4-നു ജനിച്ചു. 1836 ജൂലൈ 7-നു് ഇന്ത്യയിലെത്തി. മദ്രാസ് പ്രസിഡൻസിയുടെ വിവിധഭാഗങ്ങളിൽ മതപ്രചരണ സംബന്ധമായ ജോലികൾ നടത്തുന്നതിനിടയിൽ 1838 ഒക്ടോബർ 7-നു് ഗുണ്ടർട്ടും ഭാര്യയും തിരുനെൽവേലിയിൽ നിന്നും തിരുവന്തപുരത്തെത്തി താമസമാക്കി  \footnote{വിക്കിലേഖനം \url{ml.wikipedia.org}} .