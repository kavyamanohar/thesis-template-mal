\documentclass[12pt]{report}

\usepackage{malayalamreport}

\title{മലയാളം റിപ്പോർട്ട്}
\author{ഗവേഷക}
\date{\today}


\begin{document}
\pagenumbering{roman}
\maketitle


\center
\textbf{\LARGE\gentium{CERTIFICATE}}
\vspace{.1in}\newline


\
{\gentium This is to certify that this thesis} {\gayathri'മലയാളത്തിലെ റിപ്പോർട്ട് '} {\gentium submitted by \textbf{Researcher Name}for the award of the degree of Doctor of Philosophy in the Department of Malayalam, Kerala University is a record of bonafied research carried out under my supervision.

\vspace{.5in}
\flushright
Dr. Name of Guide
\flushleft
Place$ :$ Kariavattom
\newline
Date$  :$\hspace*{6.4cm}
}

\newpage

\newpage
\thispagestyle{empty}

\begin{center}

\huge{കൃതജ്ഞത}
\\[0.5cm]

\end{center}

\normalsize നന്ദി \\[1.0cm]


\vspace{.5in}


% Bottom of the page
\begin{flushright}
ഗവേഷക

\end{flushright}

\begin{flushleft}
Kariavattom\\
Date:
\end{flushleft}


\tableofcontents

\pagenumbering{arabic}

\chapter{ആമുഖം}
ലാടെൿ ഉപയോഗിച്ച് തയ്യാറാക്കുന്ന മലയാളം തീസിസ് റിപ്പോർട്ടിന് ഒരു ആമുഖം.

\chapter{യൂണിക്കോഡും മലയാളവും}
ലാടെൿ ഉപയോഗിച്ച് തയ്യാറാക്കുന്ന മലയാളം തീസിസ് റിപ്പോർട്ടിലെ അടുത്ത അദ്ധ്യായം.

\section{ഒന്നാം സെക്ഷൻ}

ഇത് ഒന്നാമത്തെ സെൿഷൻ.

\subsection{ഒന്നാം സബ്സെക്ഷൻ}
ഇത് സബ് സെക്ഷൻ


\chapter{ചിത്രങ്ങൾ ചേർക്കാം}
ലാടെൿ ഉപയോഗിച്ച് തയ്യാറാക്കുന്ന മലയാളം തീസിസ് റിപ്പോർട്ടിലെ ചിത്രങ്ങളുള്ള അദ്ധ്യായം

\section{ലോഗോ}

ഇത് ഒന്നാമത്തെ സെൿഷൻ. ഇവിടെ നിങ്ങൾക്ക് ചിത്രം \ref{logo} കാണാം.

\begin{figure}
\center
\includegraphics[width=0.3\textwidth]{./contents/logo.png}
\caption{കേരളസർവ്വകലാശാലയുടെ ലോഗോ}
\label{logo}
\end{figure}

\end{document}